%% LyX 1.3 created this file.  For more info, see http://www.lyx.org/.
%% Do not edit unless you really know what you are doing.
\documentclass[english]{article}
\usepackage[T1]{fontenc}
\usepackage[latin1]{inputenc}
\IfFileExists{url.sty}{\usepackage{url}}
                      {\newcommand{\url}{\texttt}}

\makeatletter

%%%%%%%%%%%%%%%%%%%%%%%%%%%%%% LyX specific LaTeX commands.
%% Bold symbol macro for standard LaTeX users
\newcommand{\boldsymbol}[1]{\mbox{\boldmath $#1$}}


%%%%%%%%%%%%%%%%%%%%%%%%%%%%%% Textclass specific LaTeX commands.
 \newenvironment{lyxcode}
   {\begin{list}{}{
     \setlength{\rightmargin}{\leftmargin}
     \setlength{\listparindent}{0pt}% needed for AMS classes
     \raggedright
     \setlength{\itemsep}{0pt}
     \setlength{\parsep}{0pt}
     \normalfont\ttfamily}%
    \item[]}
   {\end{list}}
 \newenvironment{lyxlist}[1]
   {\begin{list}{}
     {\settowidth{\labelwidth}{#1}
      \setlength{\leftmargin}{\labelwidth}
      \addtolength{\leftmargin}{\labelsep}
      \renewcommand{\makelabel}[1]{##1\hfil}}}
   {\end{list}}

\usepackage{babel}
\makeatother
\begin{document}

\title{Cells tutorial}

\maketitle
\tableofcontents{}


\section{Introduction}


\subsection{What's cells?}

Cells is a Common Lisp library that extends the language, and in particular
its object system, to let you write dataflow-driven programs. What
does this mean? This means that the flow of control of the program
depends no more on the sequence of function/method calls, but on the
data. Cells lets you specify the dependence beetwen different slots%
\footnote{A slot is the Common Lisp equivalent of a class instance variable
in other languages%
} in a family of classes. Once these constraints have been registered,
the cells system will take care of them, and will recalculate a value
when some data on which it depends has changed. As a consequence,
the programmer just has to tell the system the \emph{relationship}
between the data, the burden of maintaining them true is handled automatically
by cells. 


\subsection{How could it improve your programs?}

Cells may not be the panacea of programming, but it sure helps a lot
in contexts where keeping a set of values consistent is crucial. A
particular set of applications where this is important are graphical
applications%
\footnote{See the cells-gtk project: \url{http://common-lisp.net/project/cells-gtk}%
}, where you need to maintain consistency between what the user sees
and the real values held by the program in its internal data structures.
An example is the state of the 'Cut' menu entry in an editor: it is
usually clickable when the user has selected a piece of text and not
clickable in all the other cases. In a normal application, to achieve
this behavior you would need to track all the methods and all the
user actions that could modify the region of text currently being
selected, and add activate/disactivate calls in all those places to
keep the menu entry in a consistent state. With cells, you just need
to tell the system that the state of the menu depends on the length
of the current text selection: if the length is 0 then the state is
'deactivated', else it is 'activated'. Now you can safely work on
the rest of the application ignoring the state of the menu: it will
be automatically recalculated every time the length of the current
selection varies. Moreover, everything relating to the menu entry
is placed near its definition, and not scattered across different
functions/methods.


\section{Installation}

The installation is quite simple once you have a working Common Lisp
system. Here I will assume that you've got a working copy of SBCL%
\footnote{SBCL \url{http://www.sbcl.org}%
}. First of all, download cells: you can get the latest version at
\url{http://common-lisp.net/cgi-bin/viewcvs.cgi/cells/?root=cells}.
Then enter the directory \textasciitilde{}/.sbcl/site and unpack cells:

\begin{lyxcode}
\$~cd~\textasciitilde{}/.sbcl/site

\$~tar~-zxvf~\textasciitilde{}/cells.tar.gz
\end{lyxcode}
Now be sure that ASDF will be able to find it:

\begin{lyxcode}
\$~cd~\textasciitilde{}/.sbcl/systems

\$~for~a~in~`find~\textasciitilde{}/.sbcl/site/cells/~-name~{}``{*}.asdf''`~\textbackslash{}
\begin{lyxcode}
do~ln~-sf~\$a~.~\textbackslash{}
\end{lyxcode}
done
\end{lyxcode}
After that, start SBCL and evaluate the following expressions:

\begin{lyxcode}
>~(require~:asdf)



NIL

>~(asdf:oos~'asdf:load-op~:cells)

(some~output~will~follow)
\end{lyxcode}
If everything went right cells should be up and running.


\section{Our first cells program}


\subsection{The program}

Write the following piece of code in a file named hello-cells.lisp:

\begin{lyxcode}
(defmodel~hello-cells~()

~~((num~:accessor~num~:initarg~:num~:initform~(c-in~0))

~~~(square-num~:accessor~square-num~

~~~~~~~~~~~~~~~:initform~(c?~({*}~(num~self)~(num~self))))))



(defun~hello~()

~~(let~((h~(make-instance~'hello-cells)))

~~~~(dolist~(n~'(10~20~30~40~50~60~60))

~~~~~~(setf~(num~h)~n)

~~~~~~(format~t~\char`\"{}num~is~\textasciitilde{}a~and~square-num~is~\textasciitilde{}a\textasciitilde{}\%\char`\"{}~(num~h)~(square-num~h)))))
\end{lyxcode}
Now start the SBCL interpreter in the same directory and evaluate
the following:

\begin{lyxcode}
>~(asdf:oos~'asdf:load-op~:cells)

...

>~(use-package~:cells)

T

>~(load~{}``hello-cells.lisp'')

...

T

>~(hello)

num~is~10~and~square-num~is~100

num~is~20~and~square-num~is~400

num~is~30~and~square-num~is~900

num~is~40~and~square-num~is~1600

num~is~50~and~square-num~is~2500

num~is~60~and~square-num~is~3600

num~is~60~and~square-num~is~3600

NIL
\end{lyxcode}
What happens within the function 'hello'? First, an object of type
hello-cells is created. After that the program iterates over the contents
of the list '(10 20 30 40 50 60 60), and every number is used to set
the num slot of the object h. Then the num slot is printed together
with the slot square-num. The printed value of the slot num gives
us no surprise: it has the value we gave it. This doesn't hold for
the slot square-num, though: we never gave it a value within the loop,
but it always holds the square of the slot num! This is just cells
working for us: we told the system that the relation

\begin{lyxcode}
$num*num=squarenum$
\end{lyxcode}
must hold, and every time num changes, the expression ({*} (num self)
(num self)) is re-evaluated. Note that the relation isn't a mathematical
equation: you can't change square-num and expect to find its square
root in num.


\subsection{The program line-by-line}

Lets now analyze the program. The very first line uses the construct
defmodel:

\begin{lyxcode}
(defmodel~hello-cells~()
\end{lyxcode}
defmodel is very similar to defclass and everything valid in a defclass
construct is valid within defmodel%
\footnote{defmodel is a layer built on top of defclass%
}. The main difference is that all the slots defined within it will
be tracked by cells, except slots that are explicitly declared to
be ignored by the system by specifying :cell nil in the definition.

\begin{lyxcode}
((num~:accessor~num~:initarg~:num~:initform~(c-in~0))
\end{lyxcode}
Here we define the slot num as we would do within a standard class
declaration. The difference is in its initialization expression: instead
of the number 0 we have (c-in 0). Why? (c-in <expr>) is a construct
that tells cells that the value of num may be changed, so whenever
it does change a re-evaluation of all the slots that depend on it
must be triggered. If we did just write 0 instead of (c-in 0) a runtime
error would have been raised during the execution of (setf (num h)
...). So, when a slot is writable it must be signalled to cells with
the (c-in ...) construct. This is necessary to let cells do some optimizations
like avoiding to remember dependencies on slots that will never change.
Slots initialized with c-in are usually called {}``input cells''.

\begin{lyxcode}
(square-num~:accessor~square-num~

~~~~~~~~~~~~:initform~(c?~({*}~(num~self)~(num~self))))))
\end{lyxcode}
Now we define the slot square-num. There are two things to note here:
(c? <expr>) and 'self'.The first is a construct that says: {}``To
calculate the value of square-num, evaluate the expression <expr>''.
Within (c? ...) the variable self is bound to the object itself. (c?
...) automatically tracks any dependency, in this case the dependency
on the value of num: when num changes, ({*} (num self) (num self))
will be re-evaluated. Slots initialized with c? are called {}``ruled
cells''.

\begin{lyxcode}
(let~((h~(make-instance~'hello-cells)))
\end{lyxcode}
Here we use the function (make-instance <model-name> \emph{args{*}}),
to create an object of type <model-name>, in this case hello-cells,
as we would do to instantiate a normal class. You could specify an
initial value for num now: 

\begin{lyxcode}
(let~((h~(make-instance~'hello-cells~:num~(c-in~50))))
\end{lyxcode}
Note that you \emph{must} repeat the (c-in ...) construct. This is
because the behavior of the slot (input cell, constant, ruled cell)
is decided on a per instance basis, not on a per class basis. This
means that, in our example, we could have two objects of type hello-cells,
one where the slot num is settable and one where it is has a constant
value. When an object is created, all the values of its slots are
computed for the first time, in this case the expression ({*} (num
self) (num self)) is evaluated and the value given to the slot square-num.

\begin{lyxcode}
(setf~(num~h)~n)
\end{lyxcode}
This expression sets the value of the slot num to n. This is when
cells comes into action: square-num depends on num, so ({*} (num self)
(num self)) is re-evaluated after n has changed.

\begin{lyxcode}
(format~t~\char`\"{}num~is~\textasciitilde{}a~and~square-num~is~\textasciitilde{}a\textasciitilde{}\%\char`\"{}~(num~h)~(square-num~h))
\end{lyxcode}
Finally, we print the values of the two slots and discover that the
value of square-num is correctly the square of num.

As a side note, you can reset the cells system by calling (cell-reset):

\begin{lyxcode}
>~(cells-reset)

NIL
\end{lyxcode}
This could be necessary after an error has corrupted the system and
cells doesn't seem to work correctly anymore. It's also a good practice
to reset the system before running code that uses cells.


\section{The family system}

Objects whose type have been defined using defmodel can be organized
in families. A family is a tree of model instances (\emph{not} of
model classes!) that can reference each other using the functions
(fm-other ...), (fm\textasciicircum{} ...) and others. You can specify
the family tree at object creation time passing a list of children
to the argument :kids. Alternatively, you can access the slot .kids
(automatically created by defmodel) and set it at runtime to change
the family components. .kids is, by default, a slot of type c-in,
and you can access it through the method (kids object). You can change
the .kids slot to be of a type other than c-in as you could do with
any other slot. To access the members of a family you can give them
a name with the argument :md-name and then reference them by their
name. Another way to access them is through their type: you could
say, for example, {}``give me all the successors of type my-type''.
To use these features your models must inherit from the model 'family'.
Models that inherit from family have also a .value slot associated,
which you can access through the method (value self)%
\footnote{In older releases of cells you had to use (md-value self) instead%
}. The following example shows some of these things in action:

\begin{lyxcode}
(defmodel~node~(family)

~~((val~:initform~(c-in~nil)~:initarg~:val)))



(defun~math-op-family~()

~~(let~((root

~~~~~~~~~(make-instance

~~~~~~~~~~'node

~~~~~~~~~~:val~(c?~(apply~\#'+~(mapcar~\#'val~(kids~self))))

~~~~~~~~~~:kids

~~~~~~~~~~(c?

~~~~~~~~~~~~(the-kids

~~~~~~~~~~~~~~(make-kid~'node~:md-name~:n5~:val~(c-in~5))

~~~~~~~~~~~~~~(make-kid

~~~~~~~~~~~~~~~'node

~~~~~~~~~~~~~~~:val~(c?~(apply~\#'{*}~(mapcar~\#'val~(kids~self))))

~~~~~~~~~~~~~~~:kids

~~~~~~~~~~~~~~~(c?

~~~~~~~~~~~~~~~~~(the-kids

~~~~~~~~~~~~~~~~~~(make-kid~'node~:md-name~:n7~:val~(c-in~7))

~~~~~~~~~~~~~~~~~~(make-kid~'node~:md-name~:n9~:val~(c-in~9))))))))))~~~

~~~~(format~t~\char`\"{}value~of~the~tree~is~\textasciitilde{}a\textasciitilde{}\%\char`\"{}~(val~root))

~~~~(setf~(val~(fm-other~:n7~:starting~root))~10)

~~~~(format~t~\char`\"{}new~value~of~the~tree~is~\textasciitilde{}a\textasciitilde{}\%\char`\"{}~(val~root))))~
\end{lyxcode}
Write it in a file (in this case hello-cells.lisp) and load it:

\begin{lyxcode}
>~(load~{}``hello-cells.lisp'')

T

>~(math-op-family)

value~of~the~tree~is~68

new~value~of~the~tree~is~95

NIL
\end{lyxcode}
Lets' see the most important parts of the program:

\begin{lyxcode}
(defmodel~node~(family)

~~((val~:initform~(c-in~nil)~:initarg~:val)))
\end{lyxcode}
Here we define the model node: we plan to build a family of nodes,
so we inherit from the model family. The slot val will contain the
value of the node.

\begin{lyxcode}
(make-instance~

~'node

~:val~(c?~(apply~\#'+~(mapcar~\#'val~(kids~self))))
\end{lyxcode}
Now we create the main node: its value is defined as the sum of all
its children values. To get the children list we use the method (kids
self).

\begin{lyxcode}
:kids

(c?

~~(the-kids
\end{lyxcode}
We specify the children list using the :kids argument. the-kids builds
a list of children using the following arguments. the-kids also removes
nil kids and if an argument is a list then it is flattened, e.g. (the-kids
(list k1 (list (list k2 nil) k3))) will return a list with the kids
k1, k2 and k3.

\begin{lyxcode}
(make-kid~'node~:md-name~:n5~:val~(c-in~5))
\end{lyxcode}
This is the first child of the main node: we give it a name with the
:md-name argument to reference the node through it in the future.
To create an instance of a model intended to be a child you must specify
to make-instance its parent through the argument :fm-parent. make-kid
does this for us passing self as the parent.

\begin{lyxcode}
(make-kid

~'node

~:val~(c?~(apply~\#'{*}~(mapcar~\#'val~(kids~self))))

~:kids

~(c?

~~~(the-kids

~~~~~(make-kid~'node~:md-name~:n7~:val~(c-in~7))

~~~~~(make-kid~'node~:md-name~:n9~:val~(c-in~9)))))
\end{lyxcode}
The second child of the main node has two children and its value is
the product of their values.

\begin{lyxcode}
(format~t~\char`\"{}value~of~the~tree~is~\textasciitilde{}a\textasciitilde{}\%\char`\"{}~(val~root))

(setf~(val~(fm-other~:n7~:starting~root))~10)

(format~t~\char`\"{}new~value~of~the~tree~is~\textasciitilde{}a\textasciitilde{}\%\char`\"{}~(val~root))))
\end{lyxcode}
The body of the function prints the value of the tree, and through
the output you can see that it depends correctly on the values of
its children. Then we change the value of the node named :n7 and see
that the new output has changed accordingly. (fm-other <member-name>
<starting-point>) searches the family tree starting from <starting-point>,
and returns the object named <member-name>. If it is not found, and
error is raised. <starting-point> is optional, and it defaults to
'self'. We used fm-other outside of a defmodel, so there is no self
and we must supply a starting point.


\section{Defining an observer}

Cells lets you define a function to execute immediately after a c-in
slot is modified. This function is called an {}``observer''. To
define it, use the defobserver construct:

\begin{lyxcode}
(defobserver~<slot-name>~(\&optional~(<self>~self)~

~~~~~~~~~~~~~~~~~~~~~~~~~~~~~~~~~~~~(<new-value>~old-value)

~~~~~~~~~~~~~~~~~~~~~~~~~~~~~~~~~~~~(<old-value>~new-value)

~~~~~~~~~~~~~~~~~~~~~~~~~~~~~~~~~~~~(<old-value-boundp>~old-value-boundp))

~~<function-body>)
\end{lyxcode}
This function will be executed every time the slot <slot-name> of
an object of type <model-name> is modified. <old-value> will hold
the previous value of the slot, <new-value> the new one and <old-value-boundp>
will be nil if this is the first time the slot gets a value and t
otherwise. If not given, <self>, <new-value>, <old-value> and <old-value-boundp>
will default to 'self', 'new-value', 'old-value' and 'old-value-bound-p'.
In older releases of cells defobserver was called def-c-output.

Suppose we want to log all the values that the num slot assumes: we
can do this defining an observer function. Add the following lines
to hello-cells.lisp:

\begin{lyxcode}
(defobserver~num~((self~hello-cells))

~~(format~t~{}``new~value~of~num~is:~\textasciitilde{}a\textasciitilde{}\%''~new-value))
\end{lyxcode}
Now reload the file and try running (hello) again:

\begin{lyxcode}
>~(load~{}``hello-cells.lisp'')

T

>~(hello)

new~value~of~num~is:~0

new~value~of~num~is:~10

num~is~10~and~square-num~is~100

new~value~of~num~is:~20

num~is~20~and~square-num~is~400

new~value~of~num~is:~30

num~is~30~and~square-num~is~900

new~value~of~num~is:~40

num~is~40~and~square-num~is~1600

new~value~of~num~is:~50

num~is~50~and~square-num~is~2500

new~value~of~num~is:~60

num~is~60~and~square-num~is~3600~

num~is~60~and~square-num~is~3600

NIL
\end{lyxcode}
As you can see from the output, every time we set (num h) with a different
value, the action previously defined is called. This also happens
when (num h) is initialized for the first time at object creation
time. You may have noted that when we set (num h) to 60 for the second
time, the observer function isn't called: this is because when you
set a slot to a new value that is the same (according to the function
eql) as its old one, the change isn't propagated because there is
no need to propagate it: it didn't change! 

Now look at the following piece of code: 

\begin{lyxcode}
(defmodel~str-model~()

~~((str~:accessor~str~:initform~(c-in~\char`\"{}\char`\"{})~:initarg~:str)

~~~(rev-str~:accessor~rev-str~:initform~(c?~(reverse~(str~self))))))



(defobserver~str~()

~~(format~t~\char`\"{}changed!\textasciitilde{}\%\char`\"{}))



(defun~try-str-model~()

~~(let~((s~(make-instance~'str-model)))

~~~~(dolist~(l~`(\char`\"{}Hello!\char`\"{}~\char`\"{}Bye\char`\"{}~

~~~~~~~~~~~~~~~~~,(concatenate~'string~\char`\"{}By\char`\"{}~\char`\"{}e\char`\"{})~\char`\"{}!olleH\char`\"{}))

~~~~~~(setf~(str~s)~l)

~~~~~~(format~t~\char`\"{}str~is~\textbackslash{}\char`\"{}\textasciitilde{}a\textbackslash{}\char`\"{},~rev-str~is~\textbackslash{}\char`\"{}\textasciitilde{}a\textbackslash{}\char`\"{}\textasciitilde{}\%\char`\"{}~

~~~~~~~~~~~~~~(str~s)~(rev-str~s)))))
\end{lyxcode}
It does nothing new: it constrains rev-str to be the reverse of str,
creates an instance of str-model and prints some strings together
with their reverse. It also logs every time it needs to compute the
reversed string. Note that the second and the third strings of the
list are actually equal. Lets try to run the code (supposing you wrote
it in hello-cells.lisp):

\begin{lyxcode}
>~(load~{}``hello-cells.lisp'')

T

>~(try-str-model)

changed!

changed!

str~is~\char`\"{}Hello!\char`\"{},~rev-str~is~\char`\"{}!olleH\char`\"{}

changed!

str~is~\char`\"{}Bye\char`\"{},~rev-str~is~\char`\"{}eyB\char`\"{}

changed!

str~is~\char`\"{}Bye\char`\"{},~rev-str~is~\char`\"{}eyB\char`\"{}

changed!

str~is~\char`\"{}!olleH\char`\"{},~rev-str~is~\char`\"{}Hello!\char`\"{}

NIL
\end{lyxcode}
The reversed string is calculated \emph{every} time we set (str s),
even when we're changing it from {}``Bye'' to {}``Bye''. But {}``Bye''
and {}``Bye'' are equal! Why do we need to waste time reversing
it twice? Because cells by default uses eql to test for equality and
if two strings aren't the same string (i.e. they don't have the same
memory address) eql considers them to be different. The following
piece of code shows us another problem: suppose we change 

\begin{lyxcode}
`(\char`\"{}Hello!\char`\"{}~\char`\"{}Bye\char`\"{}~,(concatenate~'string~\char`\"{}By\char`\"{}~\char`\"{}e\char`\"{})~\char`\"{}!olleH\char`\"{})
\end{lyxcode}
to

\begin{lyxcode}
`(\char`\"{}Hello!\char`\"{}~\char`\"{}Bye\char`\"{}~\char`\"{}Bye\char`\"{}~\char`\"{}!olleH\char`\"{})
\end{lyxcode}
depending on the Common Lisp implementation you run the program on
you'll have a different output! Solving the problem is easy, we just
need to use equal instead of eql as the equality function. To supply
your own equality function pass it to the :unchanged-if argument in
the slot definition:

\begin{lyxcode}
(str~:accessor~str~:initform~(c-in~\char`\"{}\char`\"{})~:initarg~:str~

~~~~~:unchanged-if~\#'equal)
\end{lyxcode}
Now we get the same expected result on any implementation:

\begin{lyxcode}
changed!

changed!

str~is~\char`\"{}Hello!\char`\"{},~rev-str~is~\char`\"{}!olleH\char`\"{}

changed!

str~is~\char`\"{}Bye\char`\"{},~rev-str~is~\char`\"{}eyB\char`\"{}

str~is~\char`\"{}Bye\char`\"{},~rev-str~is~\char`\"{}eyB\char`\"{}

changed!

str~is~\char`\"{}!olleH\char`\"{},~rev-str~is~\char`\"{}Hello!\char`\"{}

NIL
\end{lyxcode}
The equality function must accept two values: the new value of the
slot and the old one.


\section{Lazy cells}

Ruled cells are evaluated, as we have already seen, at instance creation
time and after dependent cells change. However, you may want to \emph{not}
evaluate a ruled cell until it is really needed, i.e. when the program
asks for its value. To achieve such a behavior, you can use lazy cells.
There are three types of them, depending on their laziness:

\begin{enumerate}
\item :once-asked this will get evaluated/observed on initialization, but
won't be reevaluated immediately if dependencies change, rather only
when read by application code.
\item :until-asked this does not get evaluated/observed until read by application
code, but then it becomes un-lazy, eagerly re-evaluated as soon as
any dependency changes (not waiting until asked).
\item :always this isn't evaluated/observed until read, and not reevaluated
until read after a dependency changes.
\end{enumerate}
There are two ways in which a cell can be lazy: by not being evaluated
immediately after its creation and by not responding to dependencies
change. In both cases, when the program asks for its value, the lazy
cell is evaluated (if needed). The first type embodies only the second
way, the second type only the first way and the third type is lazy
in both ways. The following example shows the behavior of lazy cells:

\begin{lyxcode}
(defmodel~lazy-test~()

~~((lazy-1~:accessor~lazy-1~:initform~(c-formula~(:lazy~:once-asked)

~~~~~~~~~~~~~~~~~~~~~~~~~~~~~~~~~~~~~~~~(append~(val~self)~(list~'!!))))

~~~(lazy-2~:accessor~lazy-2~:initform~(c\_?~(val~self)))

~~~(lazy-3~:accessor~lazy-3~:initform~(c?\_~(reverse~(val~self))))

~~~(val~:accessor~val~:initarg~:val~:initform~(c-in~nil))))



(defobserver~lazy-1~()

~~(format~t~\char`\"{}evaluating~lazy-1!\textasciitilde{}\%\char`\"{}))



(defobserver~lazy-2~()

~~(format~t~\char`\"{}evaluating~lazy-2!\textasciitilde{}\%\char`\"{}))



(defobserver~lazy-3~()

~~(format~t~\char`\"{}evaluating~lazy-3!\textasciitilde{}\%\char`\"{}))



(defun~print-lazies~(l)

~~(format~t~\char`\"{}Printing~all~the~values:\textasciitilde{}\%\char`\"{})

~~(format~t~\char`\"{}lazy-3:~\textasciitilde{}a\textasciitilde{}\%\char`\"{}~(lazy-3~l))

~~(format~t~\char`\"{}lazy-2:~\textasciitilde{}a\textasciitilde{}\%\char`\"{}~(lazy-2~l))

~~(format~t~\char`\"{}lazy-1:~\textasciitilde{}a\textasciitilde{}\%\char`\"{}~(lazy-1~l)))



(defun~try-lazies~()

~~(let~((l~(make-instance~'lazy-test~:val~(c-in~'(Im~very~lazy!)))))

~~~~(format~t~\char`\"{}Initialization~finished\textasciitilde{}\%\char`\"{})

~~~~(print-lazies~l)

~~~~(format~t~\char`\"{}Changing~val\textasciitilde{}\%\char`\"{})

~~~~(setf~(val~l)~'(who~will~be~evaluated?))

~~~~(print-lazies~l)))
\end{lyxcode}
As usual, load it and run it:

\begin{lyxcode}
>~(load~{}``hello-cells.lisp'')

T

>~(try-lazies)

evaluating~lazy-1!

Initialization~finished

Printing~all~the~values:

evaluating~lazy-3!

lazy-3:~(LAZY!~VERY~IM)

evaluating~lazy-2!

lazy-2:~(IM~VERY~LAZY!)

lazy-1:~(IM~VERY~LAZY!~!!)

Changing~val

evaluating~lazy-2!

Printing~all~the~values:

evaluating~lazy-3!

lazy-3:~(EVALUATED?~BE~WILL~WHO)

lazy-2:~(WHO~WILL~BE~EVALUATED?)

evaluating~lazy-1!

lazy-1:~(WHO~WILL~BE~EVALUATED?~!!)

NIL
\end{lyxcode}
As you can see from the code, to declare a ruled cell to be lazy you
just need to use the three constructs (c-formula (:lazy :one-asked)
...), (c\_? ...) and (c?\_ ...) for :once-asked, :until-asked and
:always lazy cells, respectively. lazy-1 is evaluated immediately,
lazy-2 and lazy-3 only when they are needed by format. After setting
(val l), on which all the lazy cells depend, lazy-2 is re-evaluated
immediately because it is of type :until-asked, while lazy-1 becomes
lazy and lazy-3 remains lazy, so these two postpone evaluation until
we ask for their values in the call to format.

As a side note, such short names may not be very easy to remember
and to read, but those constructs are so common that you'll find yourself
using them a lot, and you'll appreciate their conciseness. If you
still prefer long descriptive names, though, you can use the c-formula
construct instead of c?/c\_?/c?\_ and c-input instead of c-in (see
the {}``Functions \& macros reference'' section).


\section{Drifters}

Another type of cells are drifter cells. A drifter cell acts like
a ruled cell, but the value returned by its body is interpreted as
an increment, so after it has been re-evaluated its value becomes
its previous one plus the one returned by the body. The following
example shows drifter cells in action:

\begin{lyxcode}
(defmodel~counter~()

~~((how-many~:accessor~how-many

~~~~~~~~~~~~~:initform~(c...~(0)

~~~~~~~~~~~~~~~~~~~~~~~~~(length~(\textasciicircum{}current-elems))))

~~~(current-elems~:accessor~current-elems

~~~~~~~~~~~~~~~~~~:initform~(c-in~nil))))



(defun~try-counter~()

~~(let~((m~(make-instance~'counter)))

~~~~(dolist~(l~'((1~2~3)~(4~5)~(1~2~3~4)))

~~~~~~(setf~(current-elems~m)~l)

~~~~~~(format~t~\char`\"{}current~elements:~\textasciitilde{}\{\textasciitilde{}a~\textasciitilde{}\}\textasciitilde{}\%\char`\"{}~(current-elems~m))

~~~~~~(format~t~\char`\"{}\textasciitilde{}a~elements~seen~so~far\textasciitilde{}\%\char`\"{}~(how-many~m)))))
\end{lyxcode}
try-counter iterates other a list setting current-elems to a list
of values, and after each iteration how-many will hold the total number
of the elements within the lists seen so far. The output will be:

\begin{lyxcode}
>~(load~{}``hello-cells.lisp'')

T

>~(try-counter)

elements:~1~2~3

3~elements~seen~so~far

elements:~4~5

5~elements~seen~so~far

elements:~1~2~3~4~~

9~elements~seen~so~far

NIL
\end{lyxcode}
The important passage in the code is the initialization of how-many:

\begin{lyxcode}
(c...~(0)

~~(length~(\textasciicircum{}current-elems)))
\end{lyxcode}
(\textasciicircum{}current-elems) is just a shortcut for (current-elems
self). The construct (c... (<initial-value>) <body>) creates a drifter
cell whose initial value will be <initial-value>, in this case 0.
When current-elems changes, (length (\textasciicircum{}current-elems))
is re-evaluated, and its value is summed to how-many, so how-many
will hold the total number of elements that current-elems has held
so far.


\section{Cyclic dependencies}

It is possible to write code with cyclic dependencies: when A changes
you need to take some action that changes B, which in turn sets A,
but A has still to complete running the code needed to keep it in
a consistent state. The following code shows how this situation could
arise:

\begin{lyxcode}
(defmodel~cycle~()

~~((cycle-a~:accessor~cycle-a~:initform~(c-in~nil))

~~~(cycle-b~:accessor~cycle-b~:initform~(c-in~nil))))



(defobserver~cycle-a~()

~~(setf~(cycle-b~self)~new-value))



(defobserver~cycle-b~()

~~(setf~(cycle-a~self)~new-value))



(defun~try-cycle~()

~~(let~((m~(make-instance~'cycle)))

~~~~(setf~(cycle-a~m)~'(?~!))

~~~~(format~t~\char`\"{}\textasciitilde{}a~and~\textasciitilde{}a\char`\"{}~(cycle-a~m)~(cycle-b~m))))
\end{lyxcode}
When try-cycle sets cycle-a, its observer gets called, which sets
cycle-b which in turn sets cycle-a. This is not an infinite cycle
as it may seem, because the second time we set cycle-a we give it
the same value we gave it the first time, so the cells engine should
stop the propagation. Lets see if this does actually work:

\begin{lyxcode}
>~(load~{}``hello-cells.lisp'')

T

>~(try-cycle)

SETF~of~<2:A~CYCLE-B/NIL~=~NIL>~must~be~deferred~by~wrapping~code~in~WITH-INTEGRITY

~~~{[}Condition~of~type~SIMPLE-ERROR{]}
\end{lyxcode}
The message could vary depending on your Common Lisp implementation,
but one thing is clear: the code doesn't work. This happens because
when we set cycle-a for the second time, its observer is still running,
so cycle-a could be in an inconsistent state. The error message tells
us the solution: wrap the problematic code inside the with-integrity
construct, which makes sure that cycle-a is consistent when that piece
of code is run. The same problem exists for cycle-b and the solution
is the same. We need then to change

\begin{lyxcode}
(defobserver~cycle-a~()

~~(setf~(cycle-b~self)~new-value))
\end{lyxcode}
to

\begin{lyxcode}
(defobserver~cycle-a~()

~~(with-integrity~(:change)

~~~~(setf~(cycle-b~self)~new-value)))
\end{lyxcode}
and 

\begin{lyxcode}
(defobserver~cycle-b~()

~~(setf~(cycle-a~self)~new-value))
\end{lyxcode}
to

\begin{lyxcode}
(defobserver~cycle-b~()

~~(with-integrity~(:change)

~~~~(setf~(cycle-a~self)~new-value)))
\end{lyxcode}
Now if we reload the code and run it we'll get the correct result.
Make sure to call (cells-reset) after an error has occurred.

\begin{lyxcode}
>~(cells-reset)

NIL

>~(load~{}``hello-cells.lisp'')

T

>~(try-cycle)

(?~!)~and~(?~!)

NIL
\end{lyxcode}

\section{Synapses}

Suppose that you have a cell A that depends on another cell B, but
you want A to change only when B changes by an amount over a given
threshold, maybe because B receives data from an external probe and
you don't want A to over-react to small fluctuations. Synapses let
you do this, and they give you more control over the constraint propagation
system. Basically, using synapses you can tell the system if a change
should be propagated or not. The following example shows a {}``clock''
that changes only after a minimal amount of time:

\begin{lyxcode}
(defmodel~syn-time~()

~~((current-time~:accessor~current-time~:initarg~:current-time

~~~~~~~~~~~~~~~~~:initform~(c-in~0))

~~~(wait-time~:accessor~wait-time~:initarg~:wait-time~:initform~(c-in~0))

~~~(time-elapsed~:accessor~time-elapsed

~~~~~~~~~~~~~~~~~:initform

~~~~~~~~~~~~~~~~~(c?

~~~~~~~~~~~~~~~~~~~(f-sensitivity~:syn~((wait-time~self))

~~~~~~~~~~~~~~~~~~~~~(current-time~self))))))



(defun~try-syn-time~()

~~(let~((tm~(make-instance~'syn-time~:wait-time~(c-in~2))))

~~~~(dotimes~(n~10)

~~~~~~(format~t~\char`\"{}time~+1\textasciitilde{}\%\char`\"{})

~~~~~~(incf~(current-time~tm))

~~~~~~(format~t~\char`\"{}time-elapsed~is~\textasciitilde{}a\textasciitilde{}\%\char`\"{}~(time-elapsed~tm)))))
\end{lyxcode}
time-elapsed holds the same value of current-time, but it changes
only when current-time changes by at least wait-time units. In the
main function we simulate time with a loop that increments current-time
by one unit and then shows elapsed-time. The most important part of
the program is 

\begin{lyxcode}
(f-sensitivity~:syn~((wait-time~self))

~~(current-time~self))
\end{lyxcode}
Here we create a synapse named :syn. It is of type f-sensitivity:
(current-time self) is evaluated \emph{always}, but if the difference
between the previously propagated value (if there is one) and the
value it returns is lesser than (wait-time self), then the slot elapsed-time
won't change and, consequently, nothing will be propagated. The expected
result then will be:

\begin{lyxcode}
>~(load~{}``hello-cells.lisp'')

T

>~(try-syn-time)

time~+1

time-elapsed~is~0

time~+1

time-elapsed~is~2

time~+1

time-elapsed~is~2

time~+1

time-elapsed~is~4

time~+1

time-elapsed~is~4

time~+1

time-elapsed~is~6

time~+1

time-elapsed~is~6

time~+1

time-elapsed~is~8

time~+1

time-elapsed~is~8

time~+1

time-elapsed~is~10

NIL
\end{lyxcode}
time-elapsed changes only when the accumulated difference is at least
wait-time (2 in this case). Other synapses available are f-delta,
f-plusp, f-zerop.


\section{Example: playing sudoku}

We have seen a few example of using cells, but none of them actually
did something beside showing cells behavior. Now we will see how to
use cells to aid us resolving a sudoku puzzle. First of all, some
constants:

\begin{lyxcode}
(defparameter~{*}all-values{*}~'(1~2~3~4~5~6~7~8~9))

(defparameter~{*}row-len{*}~9)

(defparameter~{*}sq-size{*}~3)

(defparameter~{*}col-len{*}~9)
\end{lyxcode}
The input board is a vector of vectors: every position contains either
a number from 1 to 9 or a '? telling the program that it must find
a value to fill in that position. The following is an empty board: 

\begin{lyxcode}
(defparameter~{*}board{*}~

~~\#(\#(?~?~?~?~?~?~?~?~?)

~~~~\#(?~?~?~?~?~?~?~?~?)

~~~~\#(?~?~?~?~?~?~?~?~?)

~~~~\#(?~?~?~?~?~?~?~?~?)

~~~~\#(?~?~?~?~?~?~?~?~?)

~~~~\#(?~?~?~?~?~?~?~?~?)

~~~~\#(?~?~?~?~?~?~?~?~?)

~~~~\#(?~?~?~?~?~?~?~?~?)

~~~~\#(?~?~?~?~?~?~?~?~?)))
\end{lyxcode}
We will represent the playing board with the model board, which has
two slots: complete is a boolean that tells if every position on the
board has been filled with a value, and squares is a vector of vectors
representing the actual board. The slot complete is a lazy cell because
it is needed rarely and it would be a waste of time to recompute it
whenever a single square changes. Every square in the board is represented
by the model square, which has three slots: exact-val holds the actual
value of the square and it is NIL if the square is empty, possible-vals
holds a list of the values that the square could assume without conflicting
with the exact-val of the squares in the same group, and group is
a reference to an object of type square-group: squares in the same
group cannot have the same exact-val.

\begin{lyxcode}
(defmodel~square-group~()

~~((constraining~:initform~(c-in~nil)~:initarg~:constraining

~~~~~~~~~~~~~~~~~:accessor~constraining)))~\\
~\\
(defmodel~square~()

~~((group~:accessor~group~:initform~(c-in~nil))

~~~(exact-val~:accessor~exact-val~:initform~(c-in~nil)~:initarg~:exact-val)

~~~(possible-vals~

~~~~:accessor~possible-vals

~~~~:initform~

~~~~(c?

~~~~~~(when~(and~(\textasciicircum{}group)~(not~(\textasciicircum{}exact-val)))

~~~~~~~~(let~((c~(constraining~(\textasciicircum{}group))))

~~~~~~~~~~;;~a~value~must~not~be~the~same~of~the~constraining

~~~~~~~~~~(remove-if-not~

~~~~~~~~~~~\#'(lambda~(v)~

~~~~~~~~~~~~~~~(every~

~~~~~~~~~~~~~~~~\#'(lambda~(x)~

~~~~~~~~~~~~~~~~~~~~(not~(eql~v~(exact-val~x))))

~~~~~~~~~~~~~~~~c))

~~~~~~~~~~~{*}all-values{*})))))))~\\
~\\
(defun~make-square~(x)

~~(make-instance~'square~:exact-val~(c-in~(if~(eql~x~'?)~nil~x))))~\\
~\\
(defmodel~board~()

~~((complete~:accessor~complete

~~~~~~~~~~~~~:initform~

~~~~~~~~~~~~~(c?\_

~~~~~~~~~~~~~~~(every~\#'(lambda~(x)~(every~\#'exact-val~x))~(\textasciicircum{}squares))))

~~~(squares~:accessor~squares~:initarg~:squares)))~~\\
~\\
(defmethod~print-object~((self~board)~out)

~~(dotimes~(r~{*}col-len{*})

~~~~(dotimes~(c~{*}row-len{*})

~~~~~~(format~out~\char`\"{}\textasciitilde{}a~\char`\"{}~(exact-val~(at~(\textasciicircum{}squares)~r~c))))

~~~~~~(format~out~\char`\"{}\textasciitilde{}\%\char`\"{})))
\end{lyxcode}
Some helper functions to get a value from the board and to find the
next position without a value. A position is a list of two numbers:
the row and the column.

\begin{lyxcode}
(defun~at~(board~r~c)

~~(elt~(elt~board~r)~c))~\\
~\\
(defun~next-pos~(pos)

~~{}``next~position~in~the~board.~search~only~forward.~

return~NIL~if~the~board~is~finished''

~~(destructuring-bind~(r~c)~pos

~~~~(if~(=~c~(1-~{*}row-len{*}))

~~~~~~~~(if~(=~r~(1-~{*}col-len{*}))

~~~~~~~~~~~~nil

~~~~~~~~~~~~(list~(1+~r)~0))

~~~~~~~~(list~r~(1+~c)))))~\\
~\\
(defun~next-to-try~(board~pos)

~~{}``find~next~position~without~a~value~searching~forward.

return~NIL~if~there~is~none''

~~(let~((pos~(next-pos~pos)))

~~~~(when~pos

~~~~~~(let~((s~(at~board~(first~pos)~(second~pos))))

~~~~~~~~(if~(exact-val~s)

~~~~~~~~~~~~(next-to-try~board~pos)

~~~~~~~~~~~~pos)))))
\end{lyxcode}
We create a group for every square. The same square belongs to more
than one group. We put in the same group of a certain square all the
squares in the same line, in the same column or in the same block.

\begin{lyxcode}
(defun~make-groups~(squares)

~~(dotimes~(r~{*}col-len{*})

~~~~(dotimes~(c~{*}row-len{*})

~~~~~~(setf~(group~(at~squares~r~c))

~~~~~~~~~~~~(make-instance~'square-group

~~~~~~~~~~~~~~~~~~~~~~~~~~~:constraining

~~~~~~~~~~~~~~~~~~~~~~~~~~~(c-in

~~~~~~~~~~~~~~~~~~~~~~~~~~~~(delete-duplicates

~~~~~~~~~~~~~~~~~~~~~~~~~~~~~(nconc

~~~~~~~~~~~~~~~~~~~~~~~~~~~~~~(nth-col~squares~c)

~~~~~~~~~~~~~~~~~~~~~~~~~~~~~~(nth-row~squares~r)

~~~~~~~~~~~~~~~~~~~~~~~~~~~~~~(nth-block~squares~r~c)))))))))~~\\
~\\
(defun~nth-row~(board~n)

~~\char`\"{}return~row~n~of~board\char`\"{}

~~(coerce~(elt~board~n)~'list))~\\
~\\
(defun~nth-col~(board~n)

~~\char`\"{}return~column~n~of~board\char`\"{}

~~(map~'list~\#'(lambda~(x)~(elt~x~n))~board))~\\
~\\
(defun~nth-block~(board~r~c)

~~\char`\"{}return~list~of~element~in~the~same~block~of~(at~board~r~c)\char`\"{}

~~(let~((upper-left-r~({*}~{*}sq-size{*}~(floor~(/~r~{*}sq-size{*}))))

~~~~~~~~(upper-left-c~({*}~{*}sq-size{*}~(floor~(/~c~{*}sq-size{*})))))

~~~~(apply~\#'concatenate~'list~

~~~~~~~~~~~(map~'list~\#'(lambda~(x)

~~~~~~~~~~~~~~~~~~~~~~~~~~(subseq~x~upper-left-c

~~~~~~~~~~~~~~~~~~~~~~~~~~~~~~~~~~(+~upper-left-c~{*}sq-size{*})))

~~~~~~~~~~~~~~~~(subseq~board~upper-left-r~

~~~~~~~~~~~~~~~~~~~~~~~~(+~upper-left-r~{*}sq-size{*}))))))
\end{lyxcode}
To create the board we map the input board and for every position
we create a square. Then we build all the groups.

\begin{lyxcode}
(defun~make-board~(b)

~~(let~((b~(make-instance~

~~~~~~~~~~~~'board~

~~~~~~~~~~~~:squares~

~~~~~~~~~~~~(c-in

~~~~~~~~~~~~~(map~'vector~\#'(lambda~(x)~(map~'vector~\#'make-square~x))~~b)))))

~~~~(make-groups~(squares~b))

~~~~b))
\end{lyxcode}
The following code looks for a solution, trying all the possible combinations.
Thanks to how we defined possible-vals, impossible combinations are
never tried.

\begin{lyxcode}
(defun~search-solution~(b~\&optional~(next~

~~~~~~~~~~~~~~~~~~~~~~~~~~~~~~~~~~~~~(next-to-try~(squares~b)~(list~0~-1))))

~~(if~next

~~~~~~(let~((s~(at~(squares~b)~(first~next)~(second~next))))

~~~~~~~~(or~(some~;~find~the~first~of~the~possible~values~that~yields~a~solution

~~~~~~~~~~~~~\#'(lambda~(x)

~~~~~~~~~~~~~~~~~(setf~(exact-val~s)~x)~;~try~it

~~~~~~~~~~~~~~~~~(search-solution~b~(next-to-try~(squares~b)~next)))

~~~~~~~~~~~~~(possible-vals~s))

~~~~~~~~~~~~;;~couldn't~find~any~solution:~reset~the~square~and~return~

~~~~~~~~~~~~;;~NIL~to~indicate~failure

~~~~~~~~~~~~(setf~(exact-val~s)~nil)))

~~~~~~;;~tried~all~the~positions:~have~we~completed~the~board?

~~~~~~(complete~b)))
\end{lyxcode}
Finally the main function: it accepts an input board and prints a
solution.

\begin{lyxcode}
(defun~sudoku~(the-board)

~~(let~((b~(make-board~the-board)))

~~~~(search-solution~b)

~~~~(format~t~\char`\"{}Solution:\textasciitilde{}\%\textasciitilde{}a\textasciitilde{}\%\char`\"{}~b)))
\end{lyxcode}
Save it in a file named {}``sudoku.lisp'' and try it on the empty
board:

\begin{lyxcode}
>~(load~{}``sudoku.lisp'')

T

>~(sudoku~{*}board{*})

Solution:

1~2~3~4~5~6~7~8~9

4~5~6~7~8~9~1~2~3

7~8~9~1~2~3~4~5~6

2~1~4~3~6~5~8~9~7

3~6~5~8~9~7~2~1~4

8~9~7~2~1~4~3~6~5

5~3~1~6~4~2~9~7~8

6~4~2~9~7~8~5~3~1

9~7~8~5~3~1~6~4~2~

NIL
\end{lyxcode}
It does find a solution, but it takes quite a while to print it:

\begin{lyxcode}
>~(time~(sudoku~{*}board{*}))

Solution:

1~2~3~4~5~6~7~8~9~

4~5~6~7~8~9~1~2~3~

7~8~9~1~2~3~4~5~6~

2~1~4~3~6~5~8~9~7~

3~6~5~8~9~7~2~1~4~

8~9~7~2~1~4~3~6~5~

5~3~1~6~4~2~9~7~8~

6~4~2~9~7~8~5~3~1~

9~7~8~5~3~1~6~4~2~\\
~\\
Evaluation~took:

~~2.476~seconds~of~real~time

~~2.388149~seconds~of~user~run~time

~~0.076004~seconds~of~system~run~time

~~{[}Run~times~include~0.176~seconds~GC~run~time.{]}

~~0~calls~to~\%EVAL

~~0~page~faults~and

~~155,622,072~bytes~consed.

NIL
\end{lyxcode}
It takes 2.4 seconds and it allocates more than 155 MB of memory!
We can do better by noticing that when we set exact-val in search-solution
the slot possible-vals of every cell in the same group are recomputed,
but we don't need all those values immediately, and a lot of them
will change again before we will need them. The solution is to use
a lazy cell and to do that we change the initform of possible-vals
to use c?\_ instead of c?. Change it and run the program again:

\begin{lyxcode}
>~(load~{}``sudoku.lisp'')

T

>~(time~(sudoku~{*}board{*}))

Solution:

1~2~3~4~5~6~7~8~9~

4~5~6~7~8~9~1~2~3~

7~8~9~1~2~3~4~5~6~

2~1~4~3~6~5~8~9~7~

3~6~5~8~9~7~2~1~4~

8~9~7~2~1~4~3~6~5~

5~3~1~6~4~2~9~7~8~

6~4~2~9~7~8~5~3~1~

9~7~8~5~3~1~6~4~2~~\\
~\\
Evaluation~took:

~~0.181~seconds~of~real~time

~~0.16001~seconds~of~user~run~time

~~0.020001~seconds~of~system~run~time

~~{[}Run~times~include~0.012~seconds~GC~run~time.{]}

~~0~calls~to~\%EVAL

~~0~page~faults~and

~~9,517,528~bytes~consed.

NIL~
\end{lyxcode}
Now the speed is much better (more than ten times faster), it allocates
only 9.5 MB of memory, and we achieved this result with a really small
change.


\section{Functions \& macros reference}

Here follows a quick reference of the main functions and macros.


\subsection{Main}

\begin{lyxlist}{00.00.0000}
\item [defmodel]~
\end{lyxlist}
\begin{lyxcode}
(defmodel~<model-name>~(<superclass>{*})

~~(<slot-definition>{*})

~~<other-optional-arguments>)
\end{lyxcode}
(Macro) Defines a new model. It has the same structure and the accept
the same options of a class definition. <slot-definition> accepts
the special argument :cell that lets you declare what kind of slot
it is. The default is a normal cell slot. Other options include:

\begin{enumerate}
\item :cell nil the slot will be ignored by the constraints-handling system
\item :cell :ephemeral when an ephemeral slot is changed, everything works
as with a normal cell, but after the propagation has ended, its value
will become nil. They are useful to model events.
\end{enumerate}
For every cell's accessor defmodel creates a macro \textasciicircum{}<accessor-name>
that you can use as a shortcut for (<accessor-name> self).

\begin{lyxlist}{00.00.0000}
\item [c-in]~
\end{lyxlist}
\begin{lyxcode}
(c-in~<expr>)
\end{lyxcode}
(Macro) Initializes a cell slot with the value expr. When a cell slot
initialized with c-in changes, dependant cells will be recalculated.
The value of a cell slot initialized with c-in can be setted.

\begin{lyxlist}{00.00.0000}
\item [c-input]~
\end{lyxlist}
\begin{lyxcode}
(c-input~(\&rest~args)~\&optional~value)
\end{lyxcode}
(Macro) Same as c-in, but it lets specify extra arguments, and value
is optional. If it is not given, the slot will be unbound and any
access to it will result into an error. 

\begin{lyxlist}{00.00.0000}
\item [c?]~
\end{lyxlist}
\begin{lyxcode}
(c?

~~<body>)
\end{lyxcode}
(Macro) Initializes a cell slot with the value of <body>. If <body>
references input cell slots, it will be recalculated whenever those
slots change. Within c? you have access to the variable self, representing
the current object

\begin{lyxlist}{00.00.0000}
\item [c?+n]~
\end{lyxlist}
\begin{lyxcode}
(c?+n

~~<body>)
\end{lyxcode}
(Macro) Creates a ruled cell whose value can be setted.

\begin{lyxlist}{00.00.0000}
\item [c?once]~
\end{lyxlist}
\begin{lyxcode}
(c?once

~~<body>)
\end{lyxcode}
(Macro) Creates a ruled cell that gets evaluated only once at initialization
time.

\begin{lyxlist}{00.00.0000}
\item [c?1]~
\end{lyxlist}
\begin{lyxcode}
(c?1

~~<body>)
\end{lyxcode}
(Macro) Nickname for c?once.

\begin{lyxlist}{00.00.0000}
\item [c\_?]~
\end{lyxlist}
\begin{lyxcode}
(c\_?~

~~<body>)
\end{lyxcode}
(Macro) Creates a lazy ruled cell slot of type :until-asked.

\begin{lyxlist}{00.00.0000}
\item [c?\_]~
\end{lyxlist}
\begin{lyxcode}
(c?\_~

~~<body>)
\end{lyxcode}
(Macro) Creates a lazy ruled cell slot of type :always.

\begin{lyxlist}{00.00.0000}
\item [c\_1]~
\end{lyxlist}
\begin{lyxcode}
(c\_1

~~<body>)
\end{lyxcode}
(Macro) Creates a lazy cell that gets evaluated only once.

\begin{lyxlist}{00.00.0000}
\item [c...]~
\end{lyxlist}
\begin{lyxcode}
(c...~(<initial-value>)

~~<body>)
\end{lyxcode}
(Macro) Creates a drifter cell with initial value <initial-value>.

\begin{lyxlist}{00.00.0000}
\item [c-formula]~
\end{lyxlist}
\begin{lyxcode}
(c-formula~(<options>)

~~<body>)
\end{lyxcode}
(Macro) Same as c?, but lets you specify extra options. For example,
the option :inputp lets you build a cell that behaves like a cell
initialized with both c? and c-in. Another useful option is :lazy
that lets you specify the laziness of the cell: nil, t, :once-asked,
:until-asked or :always.

\begin{lyxlist}{00.00.0000}
\item [not-to-be]~
\end{lyxlist}
\begin{lyxcode}
(not-to-be~<object>)
\end{lyxcode}
(Function) Tells cells to stop handling <object>.

\begin{lyxlist}{00.00.0000}
\item [defobserver]~
\end{lyxlist}
\begin{lyxcode}
(defobserver~<slot-name>~(\&optional~(<self>~self)~

~~~~~~~~~~~~~~~~~~~~~~~~~~~~~~~~~~~~(<new-value>~old-value)

~~~~~~~~~~~~~~~~~~~~~~~~~~~~~~~~~~~~(<old-value>~new-value)

~~~~~~~~~~~~~~~~~~~~~~~~~~~~~~~~~~~~(<old-value-boundp>~old-value-boundp))

~~<function-body>)
\end{lyxcode}
(Macro) Defines a function that is called every time the slot <slot-name>
changes. In previous versions of cells it were called def-c-output.

\begin{lyxlist}{00.00.0000}
\item [with-integrity]~
\end{lyxlist}
\begin{lyxcode}
(with-integrity~(\&optional~<opcode>~<defer-info>)

~~<body>)
\end{lyxcode}
(Macro) Makes sure to run <body> only when the system is in a consistent
state. <opcode> tells what type of anomaly should be handled. Possible
values are :tell-dependents, :awaken, :client, :ephemeral-reset and
:change.

\begin{lyxlist}{00.00.0000}
\item [without-c-dependency]~
\end{lyxlist}
\begin{lyxcode}
(without-c-dependency

~~<body>)
\end{lyxcode}
(Macro) Executes body without tracing any dependency. I.e. if we are
within a ruled cell and <body> references a cell slot, when that slot
changes the change is not propagated to the ruled cell.


\subsection{Family models}

The following only works for models that inherit from family.

\begin{lyxlist}{00.00.0000}
\item [make-part]~
\end{lyxlist}
\begin{lyxcode}
(make-part~<md-name>~<model-name>~\&rest~<args>)
\end{lyxcode}
(Function) Creates an instance of <model-name> with :md-name set to
<md-name>. <args> are passed to make-instance.

\begin{lyxlist}{00.00.0000}
\item [mk-part]~
\end{lyxlist}
\begin{lyxcode}
(mk-part~<md-name>~(<model-name>)~\&rest~<args>)
\end{lyxcode}
(Macro) Same as make-part, but sets the parent to self

\begin{lyxlist}{00.00.0000}
\item [the-kids]~
\end{lyxlist}
\begin{lyxcode}
(the-kids~\&rest~<kids>)
\end{lyxcode}
(Macro) Builds a list of kids. <kids> may contain objects or nested
lists of objects.

\begin{lyxlist}{00.00.0000}
\item [make-kid]~
\end{lyxlist}
\begin{lyxcode}
(make-kid~<model-name>~\&rest~<args>)
\end{lyxcode}
(Macro) The same as (make-instance <model-name> <args> :fm-parent
self).

\begin{lyxlist}{00.00.0000}
\item [def-kid-slots]~
\end{lyxlist}
\begin{lyxcode}
(def-kid-slots~\&rest~<kids>)
\end{lyxcode}
(Macro) Creates a function of one argument that builds a list of kids
with the given parent. Within def-kid-slots self is bound to the parent. 

\begin{lyxlist}{00.00.0000}
\item [kids]~
\end{lyxlist}
\begin{lyxcode}
(kids~<object>)
\end{lyxcode}
(Method) Gives access to <object>'s children.

\begin{lyxlist}{00.00.0000}
\item [kid1,kid2,last-kid]~
\end{lyxlist}
\begin{lyxcode}
(kid1~<object>)

(kid2~<object>)

(last-kid~<object>
\end{lyxcode}
(Function) Gives access, respectively, to <object>'s first, second
and last child

\begin{lyxlist}{00.00.0000}
\item [\textasciicircum{}k1,\textasciicircum{}k2,\textasciicircum{}k-last]~
\end{lyxlist}
\begin{lyxcode}
(\textasciicircum{}k1)

(\textasciicircum{}k2)

(\textasciicircum{}k-last)
\end{lyxcode}
(Macro) Shortcuts for (kid1 self), (kid2 self) and (last-kid self)

\begin{lyxlist}{00.00.0000}
\item [fm-parent]~
\end{lyxlist}
\begin{lyxcode}
(fm-parent~\&optional~(<object>~self))
\end{lyxcode}
(Method) Gives access to <object>'s parent.

\begin{lyxlist}{00.00.0000}
\item [fm-other]~
\end{lyxlist}
\begin{lyxcode}
(fm-other~<name>~\&optional~(<starting-point>~self))
\end{lyxcode}
(Macro) Looks for an object named <name> within <starting-point>'s
family.

\begin{lyxlist}{00.00.0000}
\item [fm\textasciicircum{}]~
\end{lyxlist}
\begin{lyxcode}
(fm\textasciicircum{}~<name>~\&optional~(<starting-point>~self))
\end{lyxcode}
(Macro) Same as (fm-other <name> (fm-parent <starting-point>)), but
doesn't search <starting-point> and its children.

\begin{lyxlist}{00.00.0000}
\item [fm-kid-typed]~
\end{lyxlist}
\begin{lyxcode}
(fm-kid-typed~<self>~<type>)
\end{lyxcode}
(Function) Finds the first <self>'s child whose type is <type>.

\begin{lyxlist}{00.00.0000}
\item [fm-descendant-typed]~
\end{lyxlist}
\begin{lyxcode}
(fm-descendant-typed~<self>~<type>)
\end{lyxcode}
(Function) Finds the first descendant of <self> whose type is <type>.

\begin{lyxlist}{00.00.0000}
\item [container]~
\end{lyxlist}
\begin{lyxcode}
(container~<object>)
\end{lyxcode}
(Function) Gets <object>'s parent.

\begin{lyxlist}{00.00.0000}
\item [container-typed]~
\end{lyxlist}
\begin{lyxcode}
(container-typed~<object>~<type>)
\end{lyxcode}
(Function) Gets <object>'s first ancestor of type <type>.

\begin{lyxlist}{00.00.0000}
\item [upper]~
\end{lyxlist}
\begin{lyxcode}
(upper~<object>~\&optional~(<type>~t))
\end{lyxcode}
(Function) Same as (container-typed <object> <type>).

\begin{lyxlist}{00.00.0000}
\item [fm-ascendant-typed]~
\end{lyxlist}
\begin{lyxcode}
(fm-ascendant-typed~<parent>~<type>)
\end{lyxcode}
(Function) Gets the first ancestor of type <type>, searching from
<parent> included.

\begin{lyxlist}{00.00.0000}
\item [fm-kid-named]~
\end{lyxlist}
\begin{lyxcode}
(fm-kid-named~<self>~<name>)
\end{lyxcode}
(Function) Gets <self>'s kids whose md-name is <name>.

\begin{lyxlist}{00.00.0000}
\item [fm-ascendant-named]~
\end{lyxlist}
\begin{lyxcode}
(fm-ascendant-named~<self>~<name>)
\end{lyxcode}
(Function) Gets the first ancestor whose md-name is <name> starting
from <self> included.

\begin{lyxlist}{00.00.0000}
\item [fm-descendant-named]~
\end{lyxlist}
\begin{lyxcode}
(fm-descendant-named~<self>~<name>~\&key~(<must-find>~t))
\end{lyxcode}
(Function) Gets the first successor whose md-name is <name> starting
from <self> included. If <must-find> is nil no error is raised if
it isn't found.


\subsection{Synapses}

\begin{lyxlist}{00.00.0000}
\item [f-sensitivity]~
\end{lyxlist}
\begin{lyxcode}
(f-sensitivity~<name>~(<sensitivity>~\&optional~<subtypename>)

~~<body>)
\end{lyxcode}
(Macro) Creates a synapse named <name> that propagates changes only
when the value returned by <body> differs by at least <sensitivity>
from the value it had the last time it propagated.

\begin{lyxlist}{00.00.0000}
\item [f-delta]~
\end{lyxlist}
\begin{lyxcode}
(f-delta~<name>~(\&key~<sensitivity>~(<type>~'number))~

~~<body>)
\end{lyxcode}
(Macro) Creates a synapse named <name> that propagates changes only
when the difference between the value returned by <body> and the value
it returned the previous time is strictly greater than <sensitivity>. 


\subsection{Misc}

\begin{lyxlist}{00.00.0000}
\item [cells-reset]~
\end{lyxlist}
\begin{lyxcode}
(cells-reset)
\end{lyxcode}
(Function) Resets the system.


\section{Other resources}

This tutorial just scratched the surface of cells. You can find more
documentation about cells within the 'doc' directory in the source
tarball or by looking at the source files within the directories 'cells-test',
'tutorial' and 'Use Cases'. A general overview of cells can be found
in the file cells-manifesto.txt in the source tarball. You can also
ask questions about cells on the project's mailing list: \url{http://common-lisp.net/cgi-bin/mailman/subscribe/cells-devel}
\end{document}
